\chapter{序論}
\section{XXX}
修士論文用のTeXフォーマットです.
卒業論文でも使えます.
表紙等,適宜変更してご活用ください.
左側のスペースが広いのは最終的に冊子に閉じるためです.
また,bibtexを使用します\cite{NIPS2017_3f5ee243}.

\subsection{XXX}
\figref{fig:NIT}に示すように,eps, jpeg, pngの画像を出力可能です.
また,\tbref{tab:sample-table}に示すように,表を

\begin{figure}[htbp]
  \centering
  \includegraphics[width=0.8\columnwidth]{chap1/img/NIT_logo.eps}
	\caption{NIT}
	\label{fig:NIT}
\end{figure}

\begin{table}[H]
  \centering
  \caption{Sample Table}
  \label{tab:sample-table}
  \resizebox{0.6\columnwidth}{!}{%
      \begin{tabular}{cl}
          \multicolumn{1}{l}{時間} & 項目                \\ \hline
          0                            &  test, test, test, test, test\\
          20                           & test, test, test, test, test \\
          30                           & test, test, test, test, test \\
          35                           & test, test, test, test, test \\
          60                           & test, test, test, test, test
      \end{tabular}%
  }
\end{table}

\section{本論文の構成}
第2章では,XXXについて述べる.

第3章では,XXXについて述べる.

第4章では,XXXについて述べる.

第5章では,XXXについて述べる.

第6章では,XXXについて述べる.

