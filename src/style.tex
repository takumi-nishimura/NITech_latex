% 目次の深さ
\setcounter{tocdepth}{2}

\renewcommand{\bibname}{参考文献}
\renewcommand{\figurename}{Fig.}
\renewcommand{\tablename}{Table}

% 参照を楽にするコマンド
\def \figref #1{\figurename\ref{#1}}
\def \tbref #1{\tablename\ref{#1}}
\def \equref #1{式(\ref{#1})}
\def \chapref #1{第\ref{#1}章}
\def \appendixref #1{付録\ref{#1}}
\def \secref  #1{\ref{#1}節}
\def \subsecref  #1{\ref{#1}小節}

\def\thesis #1{\gdef\@thesis{#1}}
\def\teacher #1{\gdef\@teacher{#1}}
\def\organization #1{\gdef\@organization{#1}}

\setlength{\oddsidemargin}{9mm}
\setlength{\evensidemargin}{9mm}

\setlength{\topmargin}{-21mm}
\setlength{\headheight}{3mm}
\setlength{\headsep}{24mm}

\setlength{\textwidth}{155mm}
\setlength{\textheight}{240mm}

\setlength{\parindent}{1.1zw}
\kanjiskip=0.1zw plus .1zw minus .05zw
\renewcommand{\baselinestretch}{1.2}

\pagestyle{headings}
